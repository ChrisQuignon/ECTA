\documentclass{beamer}
\usetheme{Pittsburgh} 
%\documentclass{scrartcl}

\usepackage[utf8]{inputenc}
\usepackage{default}
\usepackage[procnames]{listings}
\usepackage{graphicx}
%\usepackage[toc,page]{appendix}
\usepackage{caption}
\usepackage{hyperref}
\usepackage{color}
%\usepackage{csvsimple}
\usepackage{float}
\usepackage[T1]{fontenc}


%Bibliogrpahy?
%\usepackage{bibentry}
%\nobibliography*
%\bibentry{ }

%Python
\definecolor{keywords}{RGB}{255,0,90}
\definecolor{comments}{RGB}{0,0,113}
\definecolor{red}{RGB}{160,0,0}
\definecolor{green}{RGB}{0,150,0}
\lstset{language=Python,
    basicstyle=\ttfamily\scriptsize,
    keywordstyle=\color{keywords},
    commentstyle=\color{comments},
    stringstyle=\color{red},
    identifierstyle=\color{green},
    breaklines = true,
    columns=fullflexible,
    %Numbering and tabs
    %numbers=left,
    %numberstyle=\tiny\color{gray},
    %stepnumber=2,
    %numbersep=1em,
    tabsize=4,
    showspaces=false,
    showstringspaces=false}


\begin{document}

\title{Predicting energy consumption of a heat pump}
\subtitle{by evolving decision trees}
\author{
  Quignon, Christophe \\
  %Familyname, Name
} 
\institute{Hochschule Bonn Rhein Sieg}
\date{\today}

\begin{frame}
\titlepage{}
\end{frame}

\begin{frame}[fragile]
\frametitle{Problem}
\framesubtitle{}

\begin{block}{Idea}
Controlling heat pumps could greatly benefit by a correct prediction of the energy consumption of the heatpump.
\end{block}
\hbox{}

\begin{block}{Data}
The Recogizer GmbH provided me a dataset of 4 sensor reading, three weather indicators and an accumulated energy consumption. The data spans the time between July 2014 to February 2015, with a total of 242 days and roughly 2.787.848 dates.
\end{block}
\end{frame}


\begin{frame}[fragile]
\frametitle{Idea}
\framesubtitle{}


\begin{block}{Basis}
In the course "Learning and Adaptivity" a random forest was trained to predict the energy consumption. The single Decisiontrees randomly split the data.\\
This randomness could benefit from a genetic optimization, called Genetic Algorithm based Random Forests (GARF) as suggested in \cite{bader2012garf}
\end{block}


\end{frame}


\begin{frame}[fragile]
\frametitle{Preprocessing}
\framesubtitle{}


\begin{figure}[H]

\begin{minipage}{.5\textwidth}
  \centering
  \includegraphics[height=0.66\linewidth]{img/LA-regpred.png}
  \caption{Input(green) and output (red) of timeshifted data for prediction.}
\end{minipage}%
\begin{minipage}{.5\textwidth}
	\begin{itemize}
	\item Normalization
	\item Correlation analysis
	\item Drop precipitation
	\item Include day of the year
	\item Subsampling
	\item Timeshifting\cite{vafaeipour2014application}
	\end{itemize}
\end{minipage}%
\end{figure}
\hbox{}
16561 samples in the training and 557 samples in the prediction set. Both with 657 "dimensions" as input and one as output.

\end{frame}


\begin{frame}[fragile]
\frametitle{Algorithm}
\framesubtitle{}

\begin{block}{Evolutionary Strategy}
based on the vehicle-assignment\\
\hbox{}
\begin{itemize}
\item Free selection choice
\item Elitism
\item Genotype is a self implemented Decisiontree
\item Tree depth limit of 60
\item Fitness is the MSE of a random subsample of the training data
\item Mutation of split and prediction values with with a Gaussian
\item One static Sigma for the split and prediction value
\item Different mutation rates for nodes and leafs
\item Crossover by inserting subtrees
\end{itemize}
\end{block}

\end{frame}


\begin{frame}[fragile]
\frametitle{Parameters}
\framesubtitle{}
\begin{block}{Selection run}
\begin{lstlisting}[language=Python]
sigmas = [0.4, 0.2, 0.1, 0.05, 0.005]
iterations = [100, 200, 400]
selections = ['2+2',  '4+16', '4+32', '4,16']
init_tree_depths = [12]
leaf_mutations = [0.1]
node_mutations = [0.8]
n_samples = [500]
#60 runs
\end{lstlisting}
\end{block}

\begin{block}{Mutation run}
\begin{lstlisting}[language=Python]
sigmas = [0.2]
iterations = [200]
selections = ['4+16']
init_tree_depths = [6, 12]
leaf_mutations = [0.4, 0.2, 0.1, 0.01]
node_mutations = [0.8, 0.6, 0.4, 0.2]
n_samples = [100, 500, 1000]
#96 runs
\end{lstlisting}
\end{block}
\end{frame}


\begin{frame}[fragile]
\frametitle{Analysis}

\begin{itemize}
\item One local minima found after a few iterations
\item No steady improvement in any of 1.000 different runs

\begin{itemize}
\item All likely parameters
\item Some unlikely parameters 
\item Regular Regression
\item Boston dataset
\item Full preprocessed data
\end{itemize}

\item Fitness value is not stable
\item Fitness value did not match the prediction value
\end{itemize}


\end{frame}


\begin{frame}[fragile]
\frametitle{Analysis}
\framesubtitle{The nasty truth...}
\begin{figure}[H]

\begin{minipage}{.5\textwidth}
  \centering
  \includegraphics[width=.8\linewidth]{img/mse_comparison.png}
  \caption{Parameters, sorted by predicted MSE}
\end{minipage}%
\begin{minipage}{.5\textwidth}
  \centering
  \includegraphics[width=.8\linewidth]{img/param_correlation.png}
  \caption{Parameter correlation.}
\end{minipage}%
\end{figure}
\hbox{}
\hbox{}
The better the fitness value, the worse the prediction.

\end{frame}


\begin{frame}[fragile]
\frametitle{Analysis}
\framesubtitle{Fixed}
\begin{figure}[H]

\begin{minipage}{.5\textwidth}
  \centering
  \includegraphics[width=.8\linewidth]{img/paramruns_mse.png}
  \caption{Parameters, sorted by predicted MSE}
\end{minipage}%
\begin{minipage}{.5\textwidth}
  \centering
  \includegraphics[width=.8\linewidth]{img/paramruns_corellation.png}
  \caption{Parameter correlation.}
\end{minipage}%
\end{figure}

\hbox{}
It's the sampling.

\end{frame}


\begin{frame}[fragile]
\frametitle{Analysis}
\framesubtitle{The Winner}
\begin{figure}[H]

\begin{minipage}{.5\textwidth}
  \centering
  \includegraphics[width=.8\linewidth]{img/0p0277-2p2.png}
  \caption{Fitness progress of the lowest final MSE 0.0277.}
\end{minipage}%
\begin{minipage}{.5\textwidth}
  \centering
  \includegraphics[width=.8\linewidth]{img/best-0p0277=0p05678-2+2.png}
  \caption{Prediction of the lowest MSE with an actual MSE of 0.0568.}
\end{minipage}%
\end{figure}

\end{frame}

\begin{frame}[fragile]
\frametitle{Analysis}
\framesubtitle{The Best}
\begin{figure}[H]

\begin{minipage}{.5\textwidth}
  \centering
  \includegraphics[width=.8\linewidth]{img/0p0537-2+2.png}
  \caption{Fitness progress of the lowest final MSE 0.0537.}
\end{minipage}%
\begin{minipage}{.5\textwidth}
  \centering
  \includegraphics[width=.8\linewidth]{img/best-0p05374=0p0019-2+2.png}
  \caption{Prediction of the lowest MSE with an actual MSE of 0.0019.}
\end{minipage}%
\end{figure}
\end{frame}


\begin{frame}[fragile]
\frametitle{Comparison}
\framesubtitle{}

\begin{figure}[H]
  \center
  \includegraphics[width=.6\linewidth]{img/LA-predict-energy--0p890.png}
  \caption{Prediction of the Random Forest.}
\end{figure}

\hbox{}
\hbox{}

\begin{quote}
\hfill Je planmäßiger der Mensch vorgeht,\\
\hfill um so wirkungsvoller trifft ihn der Zufall.\\
\end{quote}
\hfill \small{- Friedrich Dürrenmatt}

\end{frame}


%BIBLIOGRPAHY?
\begin{frame}
\frametitle{Reference}

\bibliographystyle{plain}%amsalpha
\bibliography{bib.bib}

\end{frame}

%CONTENTS
%NOTES

%\begin{frame}[fragile, allowframebreaks]
%\frametitle{}
%\framesubtitle{}
%give your initials so we know whom to bug
%tabulars and long boring stuff is in \scriptsize

%\end{frame}


%COPY AND PASTE FROM HERE

%\begin{enumerate}
% \item 
%\end{enumerate}

%\hyperref{link}{text}

%\lstinputlisting[Language=Python]{ }




%\begin{figure}[H]
%  \center
%  \includegraphics[height=0.33\linewidth]{img/LA-regpred.png}
%  \caption{Correlation of the datasets.}
%  \label{fig:}
%\end{figure}

%\begin{figure}[H]
%\centering
%\begin{minipage}{.5\textwidth}
%  \centering
%  \includegraphics[height=0.6\linewidth]{img/LA-regpred.png}
%  %\caption{}
%  %\label{fig:}
%\end{minipage}%
%\begin{minipage}{.5\textwidth}
%  \centering
%  \includegraphics[height=0.6\linewidth]{img/LA-regpred.png}
%  %\caption{}
%  %\label{fig:} 
%\end{minipage}
%%\caption{}
%\end{figure}


\end{document}
