
%% This work is distributed under the LaTeX Project Public License (LPPL)
%% ( http://www.latex-project.org/ ) version 1.3, and may be freely used,
%% distributed and modified. A copy of the LPPL, version 1.3, is included
%% in the base LaTeX documentation of all distributions of LaTeX released
%% 2003/12/01 or later.
%% Retain all contribution notices and credits.
%% ** Modified files should be clearly indicated as such, including  **
%% ** renaming them and changing author support contact information. **
%%

\documentclass[conference]{IEEEtran}

\usepackage[T1]{fontenc} 

\usepackage[utf8]{inputenc}  

\usepackage[pdftex]{graphicx} 

\graphicspath{{img/}} 

\DeclareGraphicsExtensions{.pdf,.eps,.mps, .png}  

\usepackage[caption=false,font=footnotesize]{subfig} 

% *** MISC UTILITY PACKAGES ***
%
%\usepackage{ifpdf}
% Heiko Oberdiek's ifpdf.sty is very useful if you need conditional
% compilation based on whether the output is pdf or dvi.
% usage:
% \ifpdf
%   % pdf code
% \else
%   % dvi code
% \fi
% The latest version of ifpdf.sty can be obtained from:
% http://www.ctan.org/tex-archive/macros/latex/contrib/oberdiek/
% Also, note that IEEEtran.cls V1.7 and later provides a builtin
% \ifCLASSINFOpdf conditional that works the same way.
% When switching from latex to pdflatex and vice-versa, the compiler may
% have to be run twice to clear warning/error messages.

% *** CITATION PACKAGES ***
%
%\usepackage{cite}
% \cite{} output to follow that of IEEE. Loading the cite package will

% *** GRAPHICS RELATED PACKAGES ***
%
%\ifCLASSINFOpdf
%  \usepackage[pdftex]{graphicx}
  % declare the path(s) where your graphic files are
%  \graphicspath{{/img}}
  % and their extensions so you won't have to specify these with
  % every instance of \includegraphics
%  \DeclareGraphicsExtensions{.png}
%\else
  % or other class option (dvipsone, dvipdf, if not using dvips). graphicx
  % will default to the driver specified in the system graphics.cfg if no
  % driver is specified.
%  \usepackage[dvips]{graphicx}
  % declare the path(s) where your graphic files are
%  \graphicspath{{../eps/}}
  % and their extensions so you won't have to specify these with
  % every instance of \includegraphics
%  \DeclareGraphicsExtensions{.eps}
%\fi

% *** MATH PACKAGES ***
%
%\usepackage[cmex10]{amsmath}
% Also, note that the amsmath package sets \interdisplaylinepenalty to 10000
% thus preventing page breaks from occurring within multiline equations. Use:
%\interdisplaylinepenalty=2500

% *** SPECIALIZED LIST PACKAGES ***
%
%\usepackage{algorithmic}

% *** ALIGNMENT PACKAGES ***
%
%\usepackage{array}

%\usepackage{mdwmath}
%\usepackage{mdwtab}

%\usepackage{eqparbox}

% *** SUBFIGURE PACKAGES ***
%\usepackage[tight,footnotesize]{F}

%\usepackage[caption=false]{caption}
%\usepackage[sfont=footnotesize]{subfig}
%\usepackage[caption=false,font=footnotesize]{subfig}

% *** FLOAT PACKAGES ***
%
%\usepackage{fixltx2e}

%\usepackage{stfloats}
%\begin{figure*}[!b]" is not normally possible in
% LaTeX2e It also provides a command:
%\fnbelowfloat

% http://www.ctan.org/tex-archive/macros/latex/contrib/sttools/
% Documentation is contained in the stfloats.sty comments as well as in the
% presfull.pdf file. Do not use the stfloats baselinefloat ability as IEEE
% does not allow \baselineskip to stretch.

% *** PDF, URL AND HYPERLINK PACKAGES ***
%
%\usepackage{url}
% \url{my_url_here}.


% *** Do not adjust lengths that control margins, column widths, etc. ***
% *** Do not use packages that alter fonts (such as pslatex).         ***
% There should be no need to do such things with IEEEtran.cls V1.6 and later.
% (Unless specifically asked to do so by the journal or conference you plan
% to submit to, of course. )


% correct bad hyphenation here
\hyphenation{}


\begin{document}
%
% paper title
% can use linebreaks \\ within to get better formatting as desired
\title{Predicting energy consumption of a heat pump
by evolving decision trees.}


% author names and affiliations
% use a multiple column layout for up to three different
% affiliations
\author{\IEEEauthorblockN{Christophe Quignon}
\IEEEauthorblockA{Bonn-Rhein-Sieg University of Applied Sciences\\
Email: christophe.quignon@smail.inf.h-brs.de}
}

% make the title area
\maketitle


\begin{abstract}
In this project, I explored the capabilities of evolving decision trees to predict the energy consumption of a heat pump. Data from a real heat pump was recorded in a timespan of 242 days. From this data approximately 16.561 sample sets of input an output where used to evolve a decision tree. After that 557 samples (8 days) where used to validate the prediction.\\
The algorithm evolved a decision tree by mutation either the split value of a node or the predicted outcome. As crossover method, a node or leaf of one tree was replaced by a node or leaf of another tree.\\
Despite a wide range of parameters, it was not possible to gain a steady improvement of prediction. 
\end{abstract}
\IEEEpeerreviewmaketitle



\section{Problem}
%general paragraph on heat pumps
%what do we want to do
%choice of algorithm

\subsection{Former work}
%what was done in LA?

\subsection{Data}
%Listing of data given and its general 

\subsection{Evolving decision trees}
%short summary of decision trees
%Why evolution and not genetics?
%Link to GARF
%own decision tree implementation

\section{Preprocessing}
%normalization
%precipitation
%DOY

\subsection{Sampling}
%Subssampling and its implication on prediction, samples, dimensionality and 

\subsection{Timeshift}
%general idea of maximum data use
%use of weather forecast
%time delay of the system itself.
%input frame, features and sampling rate
%output frame, features and sampling rate

\subsection{Correlation}
%two subsets of relations (internal and external)
%Feature merge not done (temperature delta)


\section{Algorithm}
%Selection choice
%Elitism
%Tree depth
%

\subsection{Genome}
%decision tree
%leaf
%node
%genetic programming


\subsection{Mutation}
%KISS
%gaussian mutation with one sigma
%node: shift split
%leaft: change prediction

\subsection{Selection}
%free selection choice
%, does not go well
%at least two stable members
%memory use is scaling with pop size and sampling rate

\subsection{Fitness}
%mse to real values
%sampling and its impact
%random subsampling
%instable mse



\section{Analysis}
\subsection{Parameters}
%list of all parameters
%two standard sets
%some less likely parameters
%scaling of parameters with performance

\subsection{Algorithm behaviour}
%one or two local minima
%no steady improvement
%mayor problem, thus more testruns
%propaly the insteady fitness
%predicted MSE vs. fitness

\subsection{Correlation}
%which parameter impacted toe fitness
%bad sumsampling

\subsection{Comparison}
%not fruitful
%however, here is the full random forest
%comparison not fair:
%decision trees are by definition weaker then random forests
%other datasets and problems(regression) showed same behaviour


\section{Conclusion}
%not useful
%does not scale
%MSE has to be more stable

%improvement suggestions:
%more performant algorithms
%Other crossover methods
%As GP: more functions then <=
%evolve a forest instead of a tree
%include weather forecast
%personal note: I learned a lot.

%TIMESHIFT
\begin{figure}[!t] 
\centering 
\includegraphics[width=0.45\textwidth]{LA-regpred.png} 
\caption{Input(green) and output (red) of timeshifted data for prediction.} 
\label{fig:timeshift} 
\end{figure}

%MSE COMPARISON
\begin{figure}[!t] 
\centering 
\includegraphics[width=0.45\textwidth]{mse_comparison.png} 
\caption{Comparison of the general MSE (fitness) and the prediction MSE.} 
\label{fig:mse_comparison} 
\end{figure}

%PARAM CORRELATION
\begin{figure}[!t] 
\centering 
\includegraphics[width=0.45\textwidth]{paramruns_corellation.png} 
\caption{Parameter correlation.} 
\label{fig:param_correlation} 
\end{figure}

%WINNER
\begin{figure*}[!t]
\centerline{\subfloat[Fitness progress of the lowest final MSE 0.0277.]{\includegraphics[width=2.5in]{0p0277-2p2.png}%
\label{fig:winner_fitness}}
\hfil
\subfloat[Prediction of the lowest MSE with an actual MSE of 0.0568.]{\includegraphics[width=2.5in]{best-0p0277=0p05678-2+2.png}%
\label{fig:winner_prediction}}}
\caption{Analysis of the lowest final MSE of all runs.}
\label{fig:winner}
\end{figure*}


%BEST
\begin{figure*}[!t]
\centerline{\subfloat[Fitness progress of the lowest final MSE 0.0537.]{\includegraphics[width=2.5in]{0p0537-2+2.png}%
\label{fig:best_fitness}}
\hfil
\subfloat[Prediction of the lowest MSE with an actual MSE of 0.0019.]{\includegraphics[width=2.5in]{best-0p05374=0p0019-2+2.png}%
\label{fig:best_prediction}}}
\caption{Analysis of the lowest predicted MSE of all runs.}
\label{fig:best}
\end{figure*}


%RANDOM FOREST PREDICTION
\begin{figure}[!t] 
\centering 
\includegraphics[width=0.45\textwidth]{LA-predict-energy--0p890.png} 
\caption{Prediction of the Random Forest.} 
\label{fig:random_forest} 
\end{figure}


% An example of a floating table. Note that, for IEEE style tables, the 
% \caption command should come BEFORE the table. Table text will default to
% \footnotesize as IEEE normally uses this smaller font for tables.
% The \label must come after \caption as always.
%
%\begin{table}[!t]
%% increase table row spacing, adjust to taste
%\renewcommand{\arraystretch}{1.3}
% if using array.sty, it might be a good idea to tweak the value of
% \extrarowheight as needed to properly center the text within the cells
%\caption{An Example of a Table}
%\label{table_example}
%\centering
%% Some packages, such as MDW tools, offer better commands for making tables
%% than the plain LaTeX2e tabular which is used here.
%\begin{tabular}{|c||c|}
%\hline
%One & Two\\
%\hline
%Three & Four\\
%\hline
%\end{tabular}
%\end{table}





% trigger a \newpage just before the given reference
% number - used to balance the columns on the last page
% adjust value as needed - may need to be readjusted if
% the document is modified later
%\IEEEtriggeratref{8}
% The "triggered" command can be changed if desired:
%\IEEEtriggercmd{\enlargethispage{-5in}}

% references section

% can use a bibliography generated by BibTeX as a .bbl file
% BibTeX documentation can be easily obtained at:
% http://www.ctan.org/tex-archive/biblio/bibtex/contrib/doc/
% The IEEEtran BibTeX style support page is at:
% http://www.michaelshell.org/tex/ieeetran/bibtex/
%\bibliographystyle{IEEEtran}
% argument is your BibTeX string definitions and bibliography database(s)
%\bibliography{IEEEabrv, bib.bib}
%
% <OR> manually copy in the resultant .bbl file
% set second argument of \begin to the number of references
% (used to reserve space for the reference number labels box)
%\begin{thebibliography}{1}

%\bibitem{IEEEhowto:kopka}
%H.~Kopka and P.~W. Daly, \emph{A Guide to \LaTeX}, 3rd~ed.\hskip 1em plus
%  0.5em minus 0.4em\relax Harlow, England: Addison-Wesley, 1999.

%\end{thebibliography}




% that's all folks
\end{document}


